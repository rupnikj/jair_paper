\documentclass{article} % For LaTeX2e
\usepackage{nips11submit_e,times,url,comment,amsmath}
%\documentstyle[nips10submit_09,times,art10]{article} % For LaTeX 2.09



\title{Low-Rank Approximations for Large, Multi-lingual Data}

\begin{comment}
\author{
Jan Rupnik \\
A.I. Laboratory\\
Jo\v zef Stefan Institute\\
Ljubljana, Slovenia\\
\texttt{jan.rupnik@ijs.si} \\
\And
Andrej Muhi\v c\\
A.I. Laboratory\\
Jo\v zef Stefan Institute\\
Ljubljana, Slovenia\\
\texttt{andrej.muhic@ijs.si} \\
\AND
Primo\v z \v Skraba \\
A.I. Laboratory\\
Jo\v zef Stefan Institute\\
Ljubljana, Slovenia\\
\texttt{primoz.skraba@ijs.si} \\
}
\end{comment}
\author{
Jan Rupnik, Andrej Muhi\v c, Primo\v z \v Skraba \\
A.I. Laboratory\\
Jo\v zef Stefan Institute\\
Ljubljana, Slovenia\\
\texttt{(jan.rupnik,andrej.muhic,primoz.skraba@ijs.si} \\
}
% The \author macro works with any number of authors. There are two commands
% used to separate the names and addresses of multiple authors: \And and \AND.
%
% Using \And between authors leaves it to \LaTeX{} to determine where to break
% the lines. Using \AND forces a linebreak at that point. So, if \LaTeX{}
% puts 3 of 4 authors names on the first line, and the last on the second
% line, try using \AND instead of \And before the third author name.

\newcommand{\fix}{\marginpar{FIX}}
\newcommand{\new}{\marginpar{NEW}}

\nipsfinalcopy % Uncomment for camera-ready version

\begin{document}


\maketitle

\begin{abstract}
In this paper we compare low rank approximation methods for data
with a particular structure: documents in multiple
languages. Rather than looking at only 2 languages in time, we
examine the structure in up to 21 languages. The algorithms we
choose to compare are $k$-means, cross-lingual latent semantic
indexing(CL-LSI), and multi-view cannonical correlation analysis
(mCCA). We test these methods on the European Parliament
Proceedings Parallel Corpus.
\end{abstract}



\section{Introduction}
When extracting topics from documents in multiple languages, we
would like to find topics that are not only important within a
language but have an corresponding and equivalent representation
in the other languages. This could be considered a
language-independent measure of importance. Further, by finding a
low-rank approximation which is valid over multiple languages, we
can use well-established machine learning tools designed for
monolingual tasks.  Document collections are a typically
represented as high dimensional data sets, making them a prime
example for lower dimensional representations.


Our work focuses on studying non-probabilistic approaches to
multi-lingual dimensionality reduction. The methods we compare
are described in section~\ref{sec:alg}. Recently a probabilistic
approach based on latent Dirichlet allocation was proposed
\cite{xLDA}, however no implementations were readily available. We plan to implement it and include it in subsequent studies.

The paper is organized as follows: we first introduce the setting
of multilingual data, then describe the algorithms and
datasets. We conclude with the evaluation and discussion.
\begin{comment}
Concepts and... are important across countries. However, much of
what we do is still determined by language. While the Internet
originally had English content almost exclusively, now there is a
diversity of different languages well beyond bi-lingual into
truly multi-lingual territory.

The goal when doing dimensionality reduction on data is to
capture the ``important parts'' of an underlying dataset. These
will preserve the properties of the original high dimensional
data, but faster to deal with algorithmically and highlight the
components (or mixtures thereof) which best explain the data
making it useful for interpretation during analysis as well.

metric analysis

The most common representation of the

metric space....

multilingual is important....


Initially distance of kernel similarity matrices are not only
have an approximate low rank structure, but for this study we
look at matrix with additional structure.

We can view our


 Special structure of multi-lingual data
\end{comment}
\section{Multilingual Data}

Our data is a collection of documents in multiple languages along
with an alignment with correspondences across
languages. Individual documents are represented by a vector $d$
indexed by the terms of a dictionary (the $i$-th element is the
term frequency($TF$) in the document). For a corpus, we gather
document vectors into the term-document matrix $D$.  We choose to
index the columns by documents and the rows by terms $D = (d_1 ,
\ldots , d_m )$. Each document is additionally identified by its
corresponding language. Each language has its own dictionary
independent of other languages.
%The individual
%term-document matrices are grouped according to language giving $
%\{ D_1, D_2, \ldots, D_\ell\}$.

Primarily, we deal with similarities between documents. Since
each language has an independent dictionary, we only define
similarity within a language. We note that given a
transformation function between dictionaries, we could compute
similarities between documents in different languages. However,
as translations of individual words is not tailored to the corpus
in question and is often not well defined (we often have
synonyms, approximate equivalents, etc.), we will use document
correspondences to compute document similarities across
languages. Note that this implies a correspondence between
columns in each $D_k$ for all $k$. Translations between
dictionaries would give us row-based correspondences, but we will
investigate this dimension in future work.

Terms in the dictionary are generally not equally important in
determining similarity between documents, so we must preprocess
the documents. We first use term frequency ($TF$), to prune away
infrequent terms. Rather than take a fixed number of top terms in
each document we use an adaptive measure. Let $f(n)$ be a map
which returns numbers of terms appearing at least $n$ times.  We
find $n$ such that
\begin{equation}
\frac{|f(n+1) - f(n)|}{\mbox{original \# of terms}} < 0.001
\end{equation}
and used these as the terms for the document. Once this pruning
step is complete, we further re-weight the remaining. A term
weight should correspond to the importance of the term for the
given corpus.  The common weighting scheme is called Term
Frequency Inverse Document Frequency (TFIDF) weighting. A
Inverse Document Frequency (IDF) weight for dictionary term $j$
is defined as $w_j = \log( DF_j )$ where $DF_j$ is the number of
documents from the corpora which contain term $j$.  A document
TFIDF vector is its original vector multiplied element-wise by the
weights. The $j$-th element of a document vector is given by $
TF_j \log( DF_j )$. Finally, we re-normalize each vector to have
Euclidean norm equal to 1.


\begin{comment}
We will consider linear combinations of terms as ``concepts.'' The
magnitude of the coefficients of a term in a given combination
could be interpreted as the level of membership of that given
term to the concept. For the application point of view, these
could be interpreted as generalized versions of sets of
terms. Geometrically, we will interpret them as directions in the
term-space.
\end{comment}



\section{Algorithms}
\label{sec:alg}
We compute the low rank approximation of the term-document
matrix using three algorithms: $k$-means\cite{h-ca-75},
cross-lingual latent semantic indexing(CL-LSI)\cite{cl_lsi}, and
multi-view canonical correlation analysis (mCCA)\cite{Kettenring}.

$k$-means is perhaps the most well-known and used clustering
algorithm. The make the data compatible with $k$-means, we merge
all the term-document matrices into a single matrix by stacking
the individual term-document matrices.
\begin{equation}
D_{\mbox{Total}} = \begin{bmatrix}D^T_1 ,D^T_2, \cdots,D^T_\ell\
\end{bmatrix}
^T
\end{equation}
such that the columns respect the alignment of the
documents. Therefore, each document is represented by a long
vector indexed by the terms in all languages. It is these vectors
which determine the similarity on which the $k$-means is computed.

The next method is CL-LSI which is a variant of LSI~\cite{lsi}
for more than one view. Each view in this context represents a
language. It merges the individual term-document matrices in the
same way as we did for $k$-means.  LSI computes a singular value
decomposition of $D_{Total}$. Since the matrix can be large we
can use an iterative method like the Lanczos~\cite{matrix_comp}
algorithm to find the left singular vectors corresponding to the
largest singular values.

Finally, we test mCCA is specifically designed to consider data
from multiple sources (in this case languages). Therefore, we do
not merge the individual term-document matrices into one as in
the other two methods. For each language (view), we estimate the
pairwise correlation coefficients, then we try to find vectors
which maximize the sum of all pairwise correlations over all
languages. This can be written as the following optimization
\begin{equation}
\label{eq:opt1}
\max_{w_1,\ldots, w_\ell} \sum_{i=1}^\ell\sum_{j=i+1}^\ell  \frac{w_i^T D_i D_j^T w_j}{\sqrt{ w_i^T D_i D_i^T w_i\vphantom{ D_j^T } }\sqrt{ w_j^T D_j D_j^T w_j}}
\end{equation}
However, this allows only for a 1-dimensional representation
(i.e. one vector per language). Since we would like to allow for
more vectors per language, we denote the double sum in
Equation~\ref{eq:opt1} by $\mbox{mCCA}(w_1,\ldots,w_\ell)$. If
$\mbox{corr}$ denotes the correlation, to find $M$ vectors, the
optimization objective function  becomes
\begin{equation}
\max_{\substack{w_i^{(j)}\;;\;i=1,\ldots,\ell\\j=1,\ldots,M }}\sum_{s=1}^{\mbox{M}} \mbox{mCCA}\left(w_1^{(s)},\ldots,w_\ell^{(s)}\right),
\qquad\mbox{s.t.} \qquad \mbox{corr} \left(w_i^{(s)},w_i^{(t)}\right) = 0
\quad \forall s\neq t
\end{equation}
We require that the set of $M$ vectors we return for each
language are uncorrelated (so that we do not get copies
of a vector) which together maximize the pairwise correlation
between languages.


\section{Data Set}

To investigate the empirical performance of the low rank
approximations we will test the algorithms on a large-scale,
real-world multilingual dataset: the European Parliament
Proceedings Parallel Corpus v6
(EuroParl)\cite{euro_parl_web}, a corpus released by the
EU Parliament. This source offers a large number of comparable
documents in multiple languages.  In particular, EuroParl
(Release v6) provides transcripts of parliamentary session in
almost all the EU languages. This can be considered a \emph{gold
  standard} in terms of multi-lingual data: these documents are
professional translations of each other.   In the
corpus, each day is available as a separate file, although newer
data is available in smaller units. To create the document,
within each file, we create a document for each speech
ID\footnote{In the corpus, this is referred to as the speaker
  ID.}(so each speech is a document). In the end, the document
contains a document ID, a speech ID, and paragraph markup. The
documents were aligned in two ways: by speaker id and by
paragraphs.  The number of paragraphs was used only if the number
was the same for different languages.  In total, it contains
approximately 21000 documents in the 21 languages of the EU
taking up around 2GB od storage.

\begin{comment}
To investigate the empirical performance of the low rank
approximations we will test the algorithms on two large-scale,
real-world multilingual datasets: European Parliament Proceedings
Parallel Corpus v6 (EuroParl)\cite{europarl,euro_parl_web} and
Wikipedia~\cite{wikipedia}. Both sources offer a large number of
comparable documents in multiple languages.

EuroParl provides documents on legislation, regulations,
parliamentary sessions, etc.  in 21 of the EU languages and is a
\emph{gold standard} in terms of multi-lingual data: these
documents are professional translations of each other.  In the
corpus, each day is available as a separate file, although newer
data is available in smaller units. To create the document,
within each file, we create a document for each speech
ID\footnote{In the corpus, this is referred to as the speaker
  ID.}(so each speech is a document). In the end, the document
contains a document ID, a speech ID, and paragraph markup. The
documents were aligned in two ways: by speaker id and by
paragraphs.  The number of paragraphs was used only if the number
was the same for different languages.  In total, it contains
approximately 21000 documents in the 21 languages of the EU
taking up around 2GB od storage.

Wikipedia is the de-facto encyclopedia on the Internet and can be
downloaded freely. It contains 15 million articles in around 200
languages.The distribution of these articles is not uniform but
heavily biased towards a few languages. The top 3 languages
(English, German, French) contain more than 1 million articles;
the next 6 (Spanish, Italian, Dutch, Japanese, Polish, Portuguese
and Russian) contain more than 650,000 articles; down to Karuni
with 1 article. To make the two dataset comparable, we select the
same languages as above with a similar number of
documents\footnote{Of the chosen languages, Latvian had the
  fewest articles with 37,000 articles}.


\end{comment}

\section{Evaluation}
We measure the performance of the low rank approximation using
two metrics: mean average precision mate retrieval and
correlation between monolingual similarity profiles between the
query document and its nearest neighbour in the joint
representation. For each evaluation, we randomly select a
training set and test set from the data.


The first evaluation criteria we is the \emph{mean average
  precision mate retrieval score} (AMPMR). This measures the
similarity between the documents and their translations in the
common vector space induced by the latent model. Good models map
the documents close to their translations - indicating that some
language independent (semantic) information was captured.  We
evaluate each latent model (given by projection operators $P_1$
and $P_2$) by considering a pair of aligned test sets $T_1$ and
$T_2$ in languages $L_1$ and $L_2$. We select a query document
$q_1 \in T_1$ and denote the corresponding translated document
$q_2\in T_2$.  We then compute the projections $P_1 q$ and $P_2
T_2$ and rank the elements of $P_2 T_2$ by their similarity to
$P_1 q$ in the projection space (measured by Euclidean
distance). The mean average precision mate retrieval score is the
inverse of the rank of $P_2 q_2$.

This score does not give us a complete picture. The low rank of a
mate document does not necessarily indicate poor performance if
the documents which outranked it share similar content to the
query. Therefore, we compute an alternative performance measure:
\emph{correlation between monolingual similarity profiles}
(CMSP). As before, we choose a query document, target test
corpus, and project them to a common vector space. From the
target corpus, we select the closest document $r \in L_2$ to the
query in the projection space.  We then compute two similarity
profile vectors: $v_1$ contains the monolingual cosine similarity
between $q_1$ and all the documents in $T_1$ and similarly $v_2$
contains the cosine similarities between $r$ and $T_2$. The score
 is the correlation coefficient between $v_1$ and $v_2$.

The results for each of the two measures for several fitting
parameters are in Tables~\ref{tb:mean}\footnote{Updated results
  can be found in the full version of this
  paper~\cite{full_version}}. We obtain the results by averaging
over all queries in each test set, all pairs of languages and
over ten repetitions of the choice of training and test data. The
experiments are indexed by the number of training samples
(Ntrain), the number of languages we consider (NViews) and the
dimensionality of the latent space(Ndims). We also ran a
preliminary experiment on a data set based on
Wikipedia\footnote{\url{http://www.wikipedia.org}} for the same
languages.  The result is in the final row of
Table~\ref{tb:mean}.




\begin{table}
\caption{\label{tb:mean} Mean Values for Experiments}
\begin{center}
\begin{tabular}{|c|c|c|c|c|c|c|}
\hline
Parameters&\multicolumn{3}{|c|}{AMPMR} &  \multicolumn{3}{|c|}{CMSP}\\
\cline{2-7}
(Ntrain,NViews,Ndims)& $k$-means & LSI & mCCA &$k$-means & LSI & mCCA \\
\hline
\hline
(100,   3,    5) & 0.1096 & 0.2306 &0.2184& 0.1135& 0.2007 &0.1889 \\
\hline
(100,   3,   20) & 0.3906 & 0.5292 &0.5151& 0.3308& 0.4466 &0.4329 \\
\hline
(100,   10,   5) & 0.08166& 0.1870 &0.2006& 0.0833& 0.1685 &0.1725\\
\hline
(100,   10,  20) & 0.3204 & 0.4546 &0.4623& 0.2686& 0.3901 &0.3911\\
\hline
(100,   21,   5) & 0.09468& 0.2040 &0.1868& 0.1006& 0.1789 &0.1657\\
\hline
(100,  21,  20) & 0.3255 & 0.4773 &0.4577& 0.2730& 0.4076 &0.3878\\
\hline
(1000,  3,    5) & 0.1090 & 0.4059 &0.3572& 0.1162& 0.3268 &0.2837\\
\hline
(1000,  3,   20) & 0.4180 & 0.8518 &0.8748& 0.3465& 0.7725 &0.7967\\
\hline
(1000,  10,   5) & 0.08320& 0.3126 &0.3801& 0.0894& 0.2533 &0.2965\\
\hline
(1000,  10,  20) & 0.3416 & 0.7695 &0.7954& 0.2880& 0.6921 &0.7148\\
\hline
(1000,  21,   5) & 0.08427& 0.3276 &0.3187& 0.0783& 0.2618 &0.2494\\
\hline
(1000,  21,  20) & 0.3702 & 0.8002 &0.8075& 0.3110& 0.7228 &0.7240\\
\hline
\hline
(1000, 10, 10) &0.2155 & 0.2335 & 0.3169& 0.2612 & 0.2779 & 0.3362\\
\hline
\end{tabular}
\end{center}
\end{table}



\section{Discussion}

The results illustrate that both LSI and mCCA outperform
$k$-means in terms of our evaluation criteria. We believe that
this is due to mCCA and LSI capturing the word co-occurrence
patterns, a well-established fact for the LSI method, which also
holds for mCCA (The decomposition is based on the inter-lingual
covariance matrix). The performance of LSI and mCCA is comparable
in all the tested cases. This is unexpected, since LSI discards
the correspondence information between feature (word) and
language. Note that the above tables are coarse measures, as we
average over all pairs of languages. The two algorithms may have
different fail cases which may explain why this correspondence
information is does not seem important. Furthermore, in our
preliminary experiment, mCCA performed significantly better on
the Wikipedia data set. However, additional experimentation is
required before we can draw conclusions. Finally, we note that
the results show there is still room to improve the methods which
leads us to believe that a further investigation into low-rank
approximations of multilingual structure is warranted.

\bibliographystyle{unsrt} \bibliography{nips}


\pagebreak
\appendix

\section{Learned Concept Vectors}
We include the top 20 vectors we learn, each dimension 5 for
illustrative purposes. We only include 4 languages here, see
~\cite{full_version} for more examples.

\scriptsize
\begin{tabular}{p{0.5\textwidth}p{0.5\textwidth}}
\begin{verbatim}
closed written statements rule 149
women item report voted resolution
you your barroso he mr
women gender equality violence men
you young economic crisis strategy
vote you minutes details item
vote minutes details results see
you acp trade agreement european
euro countries greece acp commission
young budget treaty parliament council
acp agricultural budget agreement farmers
turkey women market accession policy
presidency trafficking violence council turkey
trafficking vote agricultural policy farmers
turkey trafficking balkans accession montenegro
item israel strategy internal trafficking
2020 strategy committee eib small
turkey acp presidency market internal
arctic products greece treaty cosmetic
arctic russia next item young
\end{verbatim}
&
\begin{verbatim}
písemná článek rozprava ukončena prohlášení
žen bodem dalším písemně usnesení
jste barroso vám vás můžete
žen násilí ženy pohlaví ženách
jste krize strategie 2020 barroso
hlasování zápis viz bodem jste
hlasování zápis viz údajů pokračujeme
dohody akt jste partnerství komise
akt komise eurozóny země řecko
parlamentu předsednictví evropského rozpočtu 20
akt dohody zemědělské klimatu zemědělství
žen turecko přistoupení turecka trhu
předsednictví násilí obchodování lidmi rady
obchodování hlasování lidmi písemně soudržnosti
makedonie obchodování turecko lidmi přistoupení
bodem strategie obchodování izrael dalším
2020 eib bodem podniky demokratů
akt turecko předsednictví turecka trh
kosmetických strategie eurozóny arktidy smlouvy
rusko bodem dalším zdraví arktidy
\end{verbatim}
\end{tabular}

\begin{tabular}{p{0.5\textwidth}p{0.5\textwidth}}
\begin{verbatim}
clos article écrites 149 débat
femmes appelle voté rapport l'ordre
vous avez barroso votre pouvez
femmes hommes l'égalité violence vous
vous jeunes crise stratégie économique
votes vote vous verbal commission
votes verbal vote procès résultats
vous acp turquie européen l'accord
euro grèce commission acp zone
jeunes budget traité parlement lisbonne
acp budget agricole pac l'accord
turquie femmes marché intérieur politique
présidence violence turquie conseil êtres
traite vote pac êtres agricole
turquie balkans occidentaux monténégro kosovo
stratégie intérieur l'ordre appelle israël
2020 stratégie entreprises bei pme
turquie acp jeunes intérieur présidence
produits cosmétiques traité zone grèce
russie arctique l'arctique jeunes santé
\end{verbatim}&
\begin{verbatim}
schriftliche geschlossen erklärungen aussprache artikel
frauen gestimmt bericht schriftlich erkläre
sie barroso herr antwort selbstverständlich
frauen gleichstellung gewalt männern geschlechter
strategie sie 2020 barroso krise
abstimmung kommission the nächster punkt
abstimmung the protokoll siehe abstimmungsstunde
akp türkei abkommen kommission staaten
eurozone griechenland akp kommission länder
lissabon parlaments parlament arbeitsmarkt vertrag
akp agrarpolitik abkommen versammlung gap
türkei frauen binnenmarkt agrarpolitik gap
gewalt menschenhandel türkei ratsvorsitz präsidentschaft
menschenhandel agrarpolitik abstimmung gap landwirte
türkei menschenhandel mazedonien montenegro kosovo
israel binnenmarkt strategie punkt menschenhandel
2020 strategie eib unternehmen kopenhagen
türkei akp binnenmarkt gewalt versammlung
arktis nanomaterialien eurozone griechenland vertrag
russland arktis arktischen punkt frau
\end{verbatim}
\end{tabular}


\end{document}

\pagebreak
\appendix

\section{Learned Concept Vectors}
We include the top 20 vectors we learn, each dimension 5 for%five most significant words for each of the multilingual topics 
illustrative purposes. We only include 4 languages here.

\scriptsize
\begin{tabular}{p{0.5\textwidth}p{0.5\textwidth}}
\begin{verbatim}
closed written statements rule 149
women item report voted resolution
you your barroso he mr
women gender equality violence men
you young economic crisis strategy
vote you minutes details item
vote minutes details results see
you acp trade agreement european
euro countries greece acp commission
young budget treaty parliament council
acp agricultural budget agreement farmers
turkey women market accession policy
presidency trafficking violence council turkey
trafficking vote agricultural policy farmers
turkey trafficking balkans accession montenegro
item israel strategy internal trafficking
2020 strategy committee eib small
turkey acp presidency market internal
arctic products greece treaty cosmetic
arctic russia next item young
\end{verbatim}
&
\begin{verbatim}
písemná článek rozprava ukončena prohlášení
žen bodem dalším písemně usnesení
jste barroso vám vás můžete
žen násilí ženy pohlaví ženách
jste krize strategie 2020 barroso
hlasování zápis viz bodem jste
hlasování zápis viz údajů pokračujeme
dohody akt jste partnerství komise
akt komise eurozóny země řecko
parlamentu předsednictví evropského rozpočtu 20
akt dohody zemědělské klimatu zemědělství
žen turecko přistoupení turecka trhu
předsednictví násilí obchodování lidmi rady
obchodování hlasování lidmi písemně soudržnosti
makedonie obchodování turecko lidmi přistoupení
bodem strategie obchodování izrael dalším
2020 eib bodem podniky demokratů
akt turecko předsednictví turecka trh
kosmetických strategie eurozóny arktidy smlouvy
rusko bodem dalším zdraví arktidy
\end{verbatim}
\end{tabular}

\begin{tabular}{p{0.5\textwidth}p{0.5\textwidth}}
\begin{tabular}{p{0.5\textwidth}p{0.5\textwidth}}
\begin{verbatim}
clos article écrites 149 débat
femmes appelle voté rapport l'ordre
vous avez barroso votre pouvez
femmes hommes l'égalité violence vous
vous jeunes crise stratégie économique
votes vote vous verbal commission
votes verbal vote procès résultats
vous acp turquie européen l'accord
euro grèce commission acp zone
jeunes budget traité parlement lisbonne
acp budget agricole pac l'accord
turquie femmes marché intérieur politique
présidence violence turquie conseil êtres
traite vote pac êtres agricole
turquie balkans occidentaux monténégro kosovo
stratégie intérieur l'ordre appelle israël
2020 stratégie entreprises bei pme
turquie acp jeunes intérieur présidence
produits cosmétiques traité zone grèce
russie arctique l'arctique jeunes santé
\end{verbatim}&
\begin{verbatim}
schriftliche geschlossen erklärungen aussprache artikel
frauen gestimmt bericht schriftlich erkläre
sie barroso herr antwort selbstverständlich
frauen gleichstellung gewalt männern geschlechter
strategie sie 2020 barroso krise
abstimmung kommission the nächster punkt
abstimmung the protokoll siehe abstimmungsstunde
akp türkei abkommen kommission staaten
eurozone griechenland akp kommission länder
lissabon parlaments parlament arbeitsmarkt vertrag
akp agrarpolitik abkommen versammlung gap
türkei frauen binnenmarkt agrarpolitik gap
gewalt menschenhandel türkei ratsvorsitz präsidentschaft
menschenhandel agrarpolitik abstimmung gap landwirte
türkei menschenhandel mazedonien montenegro kosovo
israel binnenmarkt strategie punkt menschenhandel
2020 strategie eib unternehmen kopenhagen
türkei akp binnenmarkt gewalt versammlung
arktis nanomaterialien eurozone griechenland vertrag
russland arktis arktischen punkt frau
\end{verbatim}
\end{tabular}

\begin{verbatim}
afsluttet forhandlingen skriftlige artikel erklæringer
kvinder punkt stemte vold betænkning
barroso han hr dem davies
kvinder ligestilling mænd vold kvinders
unge økonomiske 2020 barroso euroområdet
afstemningen protokollen oplysninger afstemning resultater
afstemningen protokollen resultater afstemning oplysninger
avs økonomiske tyrkiet politiske rettigheder
euroområdet grækenland lande avs forslag
unge parlamentet lissabontraktaten parlamentets budget
avs landbrugspolitik fælles aftalen ændringsforslag
tyrkiet landbrugspolitik kvinder marked fælles
formandskab menneskehandel vold tyrkiet spanske
menneskehandel landbrugspolitik afstemningen fælles områder
tyrkiet menneskehandel montenegro kosovo serbien
punkt israel menneskehandel indre marked
2020 eib små virksomheder var
tyrkiet avs unge marked indre
arktis euroområdet grækenland kosmetiske oplysninger
rusland arktis næste arktiske unge
\end{verbatim}&

\begin{tabular}{p{0.5\textwidth}p{0.5\textwidth}}
\begin{verbatim}
cierra artículo declaraciones 142 149
mujeres votado informe siguiente resolución
usted barroso le señor qué
mujeres género igualdad violencia hombres
usted crisis estrategia jóvenes barroso
votación acta véase comisión usted
votación véase acta continuamos resultados
acp acuerdo usted turquía comisión
euro países grecia comisión acp
jóvenes presupuesto parlamento tratado lisboa
acp presupuesto agrícola acuerdo pac
turquía mujeres mercado política adhesión
presidencia violencia seres turquía consejo
votación agrícola seres barroso tráfico
turquía balcanes montenegro occidentales kosovo
israel estrategia siguiente interior gaza
2020 estrategia bei pyme nombre
turquía acp mercado presidencia jóvenes
productos ártico cosméticos tratado grecia
rusia ártico ártica gas salud
\end{verbatim}&
\begin{verbatim}
kirjalikud lõppenud avaldused artikkel kodukorra
naiste päevakorrapunkt järgmine kirjalikult hääletasin
te teie olete barroso teil
naiste soolise võrdõiguslikkuse vägivalla naised
te 2020 strateegia noorte barroso
vt protokoll üksikasjad päevakorrapunkt järgmine
vt üksikasjad protokoll tulemused muud
akv türgi lepingu komisjoni te
euroala kreeka akv riikide komisjon
parlamendi lissaboni noorte eelarve oli
akv põllumajanduspoliitika eelarve ühisassamblee parlamentaarse
türgi naiste põllumajanduspoliitika soolise siseturu
nõukogu türgi inimkaubanduse vägivalla eesistujariik
põllumajanduspoliitika inimkaubanduse lissaboni barroso ühtekuuluvuspoliitika
türgi makedoonia jugoslaavia inimkaubanduse montenegro
iisraeli järgmine päevakorrapunkt palestiina inimkaubanduse
2020 eip strateegia suurusega naiste
türgi akv hispaania siseturu lepingu
arktika euroala kosmeetikatoodete lepingu kreeka
arktika venemaa päevakorrapunkt järgmine toiduohutuse
\end{verbatim}
\end{tabular}
\begin{tabular}{p{0.5\textwidth}p{0.5\textwidth}}
\begin{verbatim}
päättynyt työjärjestyksen artikla kirjalliset lausumat
naisten seuraavana esityslistalla kirjallinen laatima
te barroso puheenjohtaja olette teidän
naisten tasa naisiin sukupuolten miesten
2020 te barroso talouden talous
pöytäkirja kulkua äänestysten ks yksityiskohdat
kulkua äänestysten ks yksityiskohdat pöytäkirja
akt sopimuksen turkin euroopan valtioiden
akt kreikan valtioiden komissio kreikka
lissabonin parlamentin talousarvion nuoret neuvosto
akt yhteisen edustajakokouksen cotonoun talousarvion
turkin naisten turkki tasa 2020
puheenjohtajavaltio ihmiskaupan neuvoston väkivallan turkin
ihmiskaupan lissabonin kirjallinen alueiden viljelijöille
ihmiskaupan turkin länsi jugoslavian balkanin
israelin esityslistalla seuraavana laatima ihmiskaupan
2020 venäjän seuraavana naisten laatima
turkin akt turkki puheenjohtajavaltio sisämarkkinoita
tietojen aineiden lissabonin kreikan tärkeää
venäjän arktisen seuraavana esityslistalla venäjä
\end{verbatim}
\end{tabular}
\begin{verbatim}
rásbeli nyilatkozatok lezárom vitát szabályzat
nők rásban szavaztam napirendi pont
ön barroso úr kapni köszönöm
nők nemek férfiak erőszak egyenlőség
gazdasági ön fiatalok 2020 barroso
szavazás lásd bizottság napirendi szavazásra
lásd szavazás szavazásra jegyzőkönyvet jegyzőkönyvben
akcs európai ön közös politikai
országok akcs görögország bizottság fejlődő
fiatalok parlament lisszaboni szerződés tanács
akcs mezőgazdasági közgyűlés közös költségvetés
törökország nők közös piac belső
emberkereskedelem tanács elnökség erőszak spanyol
emberkereskedelem mezőgazdasági barroso nemzeti szavazásra
emberkereskedelem macedónia jugoszláv törökország nyugat
következő emberkereskedelem palesztin napirendi pont
2020 ebb stratégia kis következő
akcs törökország török fiatalok spanyol
kozmetikai sarkvidék szerződés északi termékek
oroszország sarkvidék északi sarkvidéki napirendi
\end{verbatim}
\end{tabular}
\begin{tabular}{p{0.5\textwidth}p{0.5\textwidth}}
\begin{verbatim}
chiusa scritte articolo dichiarazioni svolgerà
donne reca l'ordine votato relazione
lei barroso riceverà avete onorevole
donne violenza uomini genere parità
lei giovani crisi strategia barroso
votazione verbale commissione dettagliati votazioni
votazione dettagliati verbale vedasi votazioni
acp turchia lei sviluppo commissione
paesi grecia commissione donne acp
giovani bilancio trattato parlamento lisbona
acp agricola bilancio paritetica revisione
turchia donne mercato politica interno
presidenza violenza consiglio turchia esseri
votazione esseri barroso agricoltori agricola
turchia balcani macedonia iugoslava occidentali
strategia israele interno reca l'ordine
2020 strategia pmi imprese piccole
turchia acp interno presidenza giovani
prodotti cosmetici trattato grecia artica
russia artica giovani gas reca
\end{verbatim}&
\begin{verbatim}
raštiški pareiškimai straipsnis tvarkos 142
moterų balsavau moteris pranešimas raštu
jūs jūsų barroso j galite
moterų moteris lyčių vyrų lygybės
jūs euro 2020 ekonomikos darbo
protokolą balsavimo žr jūs rezultatai
žr protokolą balsavimo rezultatai išsamūs
akr europos susitarimo turkijos prekybos
euro akr komisija zonos šalių
biudžeto parlamento lisabonos jaunimo užimtumo
akr žemės ūkio klimato energijos
turkijos moterų turkija ūkio stojimo
žmonėmis tarybai moteris pirmininkaujanti turkijos
žmonėmis žemės ūkio jūs sanglaudos
žmonėmis turkijos jugoslavijos vakarų kosovo
žmonėmis miškų vidaus 2020 izraelio
2020 eib klausimas demokratų vardu
akr turkijos turkija jaunimo vidaus
kosmetikos arkties euro produktų lisabonos
arkties rusijos rusija klausimas saugos
\end{verbatim}
\end{tabular}
\begin{tabular}{p{0.5\textwidth}p{0.5\textwidth}}
\begin{verbatim}
slēgtas pants reglamenta rakstiskas deklarācijas
nākamais sieviešu balsoju punkts rakstiski
jūs jums jūsu barroso esat
sieviešu sievietēm dzimumu sievietes vīriešu
jūs 2020 eiro jauniešu darba
balsošanas jūs nākamais balsošanu protokolu
balsošanas protokolu balsošanu detalizēta sk
ākk jūs tirdzniecības eiropas nolīguma
eiro ākk jaunattīstības komisija zonas
budžeta parlamenta jauniešu lisabonas jauniešiem
ākk lauksaimniecības nolīguma klp klimata
sieviešu turcijas lauksaimniecības tirgus turcija
prezidentūras prezidentūra spānijas gada čehijas
lauksaimniecības klp jūs barroso kohēzijas
dienvidslāvijas maķedonijas turcijas rietumbalkānu melnkalne
nākamais tirgus punkts mums palestīnas
2020 eib gadam mvu budžeta
ākk turcijas turcija spānijas turciju
kosmētikas arktikas lisabonas eiro datu
arktikas krievijas nākamais gāzes arktiku
\end{verbatim}&
\begin{verbatim}
schriftelijke verklaringen gesloten artikel 149
vrouwen verslag schriftelijk gestemd verklaar
u uw bent barroso zult
vrouwen geweld mannen gendergelijkheid u
u jongeren crisis strategie economische
stemmingen u notulen commissie stemming
stemmingen notulen stemming uitslagen bijzonderheden
acs overeenkomst u turkije handel
eurozone griekenland landen commissie acs
jongeren verdrag parlement begroting raad
acs overeenkomst begroting landbouwbeleid landbouw
turkije vrouwen markt landbouwbeleid interne
voorzitterschap mensenhandel geweld raad turkije
mensenhandel landbouwbeleid landbouwers barroso gemeenschappelijk
turkije mensenhandel westelijke montenegro kosovo
israël strategie interne mensenhandel markt
2020 kleine strategie middelgrote ondernemingen
turkije acs jongeren voorzitterschap interne
eurozone arctische producten griekenland verdrag
arctische rusland orde jongeren regio
\end{verbatim}
\end{tabular}
\begin{tabular}{p{0.5\textwidth}p{0.5\textwidth}}
\begin{verbatim}
zamykam pisemne regulaminu oświadczenia art
kobiet punktem sprawozdania piśmie przemocy
pan barroso pani pana otrzyma
kobiet płci przemocy mężczyzn pan
pan 2020 euro strategii pracy
głosowanie patrz pan protokół głosowania
głosowanie patrz protokół głosowania wyniki
umowy akp komisji pan europejskiej
euro akp kraje komisja kobiet
parlamentu budżetu młodych traktatu euro
akp umowy budżetu rolnej wpr
kobiet turcji turcja rynku wpr
ludźmi prezydencji przemocy prezydencja turcji
ludźmi głosowanie barroso obszarów pan
ludźmi turcji macedonii turcja bałkanów
ludźmi punktem strategii 2020 gazy
2020 ebi kobiet małych punktem
akp turcji turcja rynku prezydencja
produktów traktatu informacji strefy danych
pani rosji gazu rosja kolejnym
\end{verbatim}&
\begin{verbatim}
encerrado escritas artigo declarações 149
mulheres segue votei relatório reaberta
barroso senhor deputado presidente presidência
mulheres igualdade violência homens género
jovens crise estratégia barroso económica
votação acta pormenorizados comissão segue
votação pormenorizados acta votações resultados
acp acordo turquia comissão comércio
euro países grécia comissão acp
jovens orçamento parlamento tratado lisboa
acp orçamento agrícola alterações paritária
turquia mulheres mercado adesão igualdade
presidência violência turquia tráfico conselho
tráfico agrícola votação seres barroso
turquia balcãs ocidentais tráfico kosovo
israel estratégia segue interno ordem
2020 estratégia pequenas bei empresas
turquia acp jovens interno mercado
árctico produtos cosméticos tratado grécia
árctico rússia jovens segue gás
\end{verbatim}
\end{tabular}
\begin{tabular}{p{0.5\textwidth}p{0.5\textwidth}}
\begin{verbatim}
închisă scrise declaraţii dezbaterea articolul
femeilor votat următorul femei scris
aţi dumneavoastră veţi barroso vă
femeilor femei bărbaţi gen egalitatea
aţi veţi muncă 2020 barroso
the comisiei ordinea următorul zi
the votarea votul vot detalii
acp aţi european turcia comisiei
euro acp ţările grecia comisia
tinerii lisabona tinerilor 20 parlamentului
acp agricolă pac bugetul acordului
turcia femeilor internă aderare piaţa
preşedinţia turcia consiliului preşedinţiei traficul
barroso traficul traficului the pac
turcia kosovo vest balcanii fosta
următorul gaza internă israelul traficul
2020 strategia 2010 bei mijlocii
turcia acp piaţa preşedinţia internă
arctică cosmetice produsele grecia lisabona
arctică rusia regiunea următorul naturale
\end{verbatim}&
\begin{verbatim}
skončila písomné článok vyhlásenia rozprava
žien alším bodom hlasoval písomne
ste barroso vám pán predseda
žien ženy násilia ženách rovnosti
ste 2020 mladých barroso krízy
ste podrobnostiach zápisnicu hlasovanie hlasovania
podrobnostiach zápisnicu pozri hlasovania hlasovanie
akt dohody ste európskej komisie
eurozóny akt krajiny komisia grécko
parlamentu mladých rozpočet zmluvy lisabonskej
akt dohody klímy spp rozpočet
žien turecko turecka trh spp
predsedníctvo ľuďmi rady predsedníctva násilia
ľuďmi hlasovanie spp barroso ste
ľuďmi balkánu macedónsko turecko pokroku
alším bodom ľuďmi izrael potrebujeme
2020 malých eib stredných podnikov
akt turecko turecka predsedníctvo trhu
kozmetických arktídu výrobkov údajov eurozóny
arktídu rusko pani bodom alším
\end{verbatim}
\end{tabular}
\begin{tabular}{p{0.5\textwidth}p{0.5\textwidth}}
\begin{verbatim}
pisne člen zaključena izjave razprava
naslednja žensk pisni obliki poročilo
ste boste barroso gospod vas
žensk ženskami spolov ženske nasilja
ste boste 2020 barroso mlade
glasovanje izide zapisnik glej ste
izide glasovanje zapisnik glej podrobnosti
akp sporazuma ste komisije evropski
akp komisija grčija gozdov države
mlade parlamenta proračun pogodbe proračuna
akp sporazuma proračun proračuna kmetijske
žensk turčija turčije trg politika
predsedstvo ljudmi sveta nasilja proračun
ljudmi glasovanje skp barroso ravni
ljudmi turčija napredku turčije jugoslovanska
točka ljudmi naslednja izrael 2020
2020 srednje eib podjetja žensk
akp turčija turčije predsedstvo mlade
kozmetičnih podatkov izdelkov pogodbe pomembno
naslednja rusija rusijo točka seveda
\end{verbatim}&
\begin{verbatim}
avslutad härmed förklaringar skriftliga debatten
kvinnor punkt röstade män våld
ni er barroso han ert
kvinnor män våld jämställdhet ni
ni ungdomar krisen ekonomiska 2020
omröstningen ni protokollet omröstningsresultat punkt
omröstningen omröstningsresultat protokollet uppgifter rör
avs ni handel avtalet europeiska
euroområdet länderna avs kommissionen grekland
ungdomar lissabonfördraget ungdomarna budget förslaget
avs gemensamma jordbrukspolitiken avtalet församlingen
turkiet kvinnor gemensamma marknaden jordbrukspolitiken
ordförandeskapet våld människohandel turkiet spanska
omröstningen jordbrukspolitiken ni människohandel gemensamma
turkiet republiken balkan f västra
punkt inre israel marknaden åtgärder
2020 små företag medelstora strategin
turkiet avs inre ordförandeskapet ungdomar
arktis produkter euroområdet nanomaterial grekland
arktis ryssland nästa ungdomar ni
\end{verbatim}
\end{tabular}

\begin{table}
\begin{center}
\caption{Standard Deviation}
\begin{tabular}{|c|c|c|c|c|c|c|}
\hline
Parameters&\multicolumn{3}{|c|}{AMPMR} &  \multicolumn{3}{|c|}{CMSP}\\
\cline{2-7}
(NViews,Ndims,Ntrain)& $k$-means & LSI & mCCA &$k$-means & LSI & mCCA \\
\hline
\hline
 &    0.0058  &   0.0100  &   0.0087  &   0.0089  &   0.0118  &   0.0098 \\
  \hline
 &    0.0046  &   0.0424  &   0.0338  &   0.0071  &   0.0341  &   0.0300 \\
  \hline
 &    0.0061  &   0.0097  &   0.0147  &   0.0081  &   0.0125  &   0.0187 \\
  \hline
 &   0.0068  &   0.0416  &   0.0381  &   0.0135  &   0.0332  &   0.0285 \\
  \hline
 &   0.0151  &   0.0093  &   0.0101  &   0.0176  &   0.0122  &   0.0137 \\
  \hline
 &    0.0045  &   0.0530  &   0.0463  &   0.0046  &   0.0458  &   0.0397 \\
\hline\hline
\end{tabular}
\end{center}
\end{table}
